\chapter{Security Policy}
\label{chap:securityPolicy}

%removed section to clean up the document.
%\section{Why is security policy needed?} % Explaining the need for a security policy, can be quite general.

%Network Security policy outlines assets which the company wants to protect.
The purpose of this policy is to establish administrative direction, requirements and guidance to ensure the protection of MjøsTaxis information on their networks.
It is essential if a company wants to protect their and their clients data, it reduces the risk of falling for data theft.
Having a security policy helps company to stay ahead of some incidents, if they happen to have a plan for it, keeping good routine while handling devices, keeping track of software updates and having guidelines for employees.

\section{Education}
Employees of MjøsTaxi are given security education dependant on their role in the company every 6 months and is provided by a 3rd party.
Education covers phishing(email, sms), good password practices, device handling, clean workspace, removable media.

\section{System Access Control}
\subsubsection*{User passwords} 
    \setlist{nolistsep}         % Removes gap between list-title and lits items.
    \begin{itemize}[noitemsep]  % noitemsep removes gap between list items.
        \item All passwords must be at least 8 characters long, contain uppercase letter and at least one number.
        \item Employees are encouraged to use passphrases.
%        \item Passwords expire every 365 days. When expired, users are required to create a new password \item that is not similar to the previous password.
        \item Passwords are not to be stored on readable notes.
        \item Employees are encouraged to use a password manager.
        \item Passwords should not be shared by employees.
        \item Every employee has proximity cards to access restricted areas or use equipment(such as printers).
    \end{itemize}{} 

\subsubsection*{Passwords on systems}
    \setlist{nolistsep}
    \begin{itemize}[noitemsep]
        \item Every device owned by Mjøsa taxi is protected by password access controls.
        \item Employees who fail to enter correct credentials five times must contact system administrator for a reset.
    \end{itemize}{}
    
\section{System Privileges}
\subsubsection*{Limiting system access}
    \setlist{nolistsep}
    \begin{itemize}[noitemsep]
        \item If an employee is inactive for longer than 15 minutes the machine will require to re-authenticate.
    \end{itemize}
\subsubsection*{Process for Granting System Privileges} 
    \setlist{nolistsep}
    \begin{itemize}[noitemsep]
        \item Employees are assigned a privilege level needed to fulfil their work.
        \item Employees who need to update, get higher privileges need to make a request which will need to be approved by system admins.
        \item Privileges granted to users that are not employees of MjøsTaxi Company will only receive them for up to 3 months and then will have to request for renewal.
    \end{itemize}

\subsubsection*{Revoking system access}
    \setlist{nolistsep}
    \begin{itemize}[noitemsep]
        \item In the event of a employee losing their job, their privileges have to be revoked before termination notice.
        \item Employees which are inactive will have their privileges revoked within 3 months.
        \item Given privileges should be reevaluated by administration every time employees are up for renewal of access privileges.
    \end{itemize}

\section{Data and Program Backup}
    \setlist{nolistsep}
    \begin{itemize}[noitemsep]
        \item Employees are required to take backup of their workstations and store on companies network servers.
        \item All confidential backups must be encrypted.
    \end{itemize}
    

\section{Portable Computers}

    \setlist{nolistsep}
    \begin{itemize}[noitemsep]
        \item Employees who have been granted laptops, notebooks, tablets and/or smartphones are responsible for devices security.
        \item Devices should not be left unattended and unlocked.
        \item Information related to the company should be encrypted.

    \end{itemize}{}
    

\section{Printing}

    \setlist{nolistsep}
    \begin{itemize}[noitemsep]
        \item Accessing printers requires identification.
        \item Printers which will be used to print confidential information should be at a restricted area.
    \end{itemize}{}
    

\section{Physical Security of Computer and Communications Gear}

    \setlist{nolistsep}
    \begin{itemize}[noitemsep]
        \item All machines and tools of Mjøsa Taxi are to be in secured locations.
        \item Working stations are off-limits for guests.

    \end{itemize}{}
    

\section{Violations}
In cases where companies policy is not followed, legal action can be taken against users in question.