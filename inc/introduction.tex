\chapter{Introduction}
\label{chap:introduction}
% Introducing the case and the company.

\section{Project Description}

\subsection{The Case}
The case is to help a company with identifying and analyzing their communcation needs, as well as designing and implementing a solution supporting these needs. 
Our case regards the taxi company, MjøsTaxi, which has three locations around Mjøsa; Gjøvik, Lillehamar and Hammer.
The three locations have 1 million, 600 000 and 400 000 inhabitants respectively.


\section{Business Case}
MjøsTaxi is a company based in Gjøvik, supplying transport services around Mjøsa. 
The company primarily focuses on taxis, but also provides maxitaxis and even bus services for bigger parties.
In addition, the company has their own garage, with mechanics providing maintenance for their vehicles.

Recent growth has enabled the company to to expand its offices to Lillehammer and Hamar as well. 
Located primarily in the city centres, the clients are people of all ages that needs a reliant way to book their transportation services.
MjøsTaxi currently has 2500 employees, 2000 of which are taxi-drivers, offering its services to a total of 2 million citizens across the three cities.

As the company is expanding, the old phone-based booking-system is being replaced with a network-based app, allowing for bookings through a web-interface, as well as a intranet application allowing employees to access and fill data about their workday. %%%% PLACEHOLDER.
Booking through phone is still a feature likely to be utilized by many customers, but will be moved over to VoIP.
The company has hired consultants to design and implement the new required infrastructure, as well as teaching the IT-department the infrastructure to ensure their ability to provide proper productivity and reliability.

The old system was limited to a call centre with 24/7 availability, where an employee's job is to take calls from potential clients, confirming a drivers availability, manually contacting available drivers, and then confirming the booking with the client. The taximeter is connected to a screen mounted inside the front of the car, as well as a terminal to process payments with a card. At the end of each shift, the drivers will add an end-of-day statement in a folder at the office. 


\subsection{Technical Service Requirements}
Our job is to implement a reliable and secure network infrastructure for the company's need of a more modern, robust, and potentially expandable network for a reasonable price. 
The network must support the following list of services.

\subsubsection{Public website}
Contains information about the company, its services, and a telephone-number. 
Booking a fare is also possible through a form on the main-page, which requires storing customer information like name, location, and contact information.

\subsubsection{Intranet}
The intranet includes a web-server to host the intranet site, which will be used by management to communicate information and news relevant to the business. 
It also includes databases to store and manage customer information, as well as accounting, logging the employees hours, and individual taximeters.
%Needs clarification ++

\subsubsection{Phone}
The main purpose of the phone-services is to take calls from customers, and booking or cancelling transportation on their behalf. The different branches must also be able to communicate with each other.

\subsubsection{Email}
Mainly for customer-service - responding to inquiries regarding booking, cancellation, etc.
As well as daily business operations.

\subsubsection{Payment}
Each taxi has wireless terminals to receive payment after a transportation. It is also possible to receive a monthly bill for the services. The terminals must be able to do local (offline) transactions if temporarily losing connection to HQ.
Local transactions should be possible, as long as the taxi is connected to the internet through their mobile connection. 
The information about the fares will then be pushed to the HQ database when the service is available again. (This needs to be done in an orderly fashion, as a longer outage could trigger an accidental DDoS when all the taxis try to sync with HQ again.)

\subsubsection{WiFi}
A standard quality of life service for employees, visitors and guests in each branch location. 
Everybody should be able to connect to the internet anywhere in the building.

\subsubsection{Redundancy}
As the business transitions over to a digital workflow, and is hosting web-services in-house, redundancy is needed.
The HQ is the only location that is operation-critical, which warrants redundant links both inside the network, as well as out to the ISP. 
The sub-locations are not as important as a network-loss will only lead to minor disruptions to the normal business-operation, because work can still be done without access to the internal application server.

\subsection{Communication needs}
Most communication in the business is handeled by the internal application server

\section{Services offered by the business}


\subsection{Users}
The users on the network can be divided into two general groups.

\subsubsection{Customers/Guests}
In this group you find customers and guests. 
This user group access the network through the wireless access-points at each location.
Users in this group will either be customers waiting for a taxi or customer support, or be people that just need some place to access the internet while they are out and about.

Customers ordering through the website also fall under this group, as they are accessing the webserver.

\subsubsection{Employees}
All remaining users are employees at the company.
This does not mean all users are treated equally in the network. 
Employees are divided into different groups, which are then put in separate VLANs.

Employee Groups
\setlist{nolistsep}         % Removes gap between list-title and lits items.
\begin{itemize}[noitemsep] 
    \item IT-Staff
    \item Tier 1
    \item Tier 2
    \item Tier 3
    \item IP Phones
\end{itemize}
\hfill

\emph{IT-staff} is in their own group, and have more access than others on the network.
They can also access a management-VLAN if need be.
\setlist{nolistsep}
\begin{itemize}[noitemsep]
    \item 2 Network Engineers
        \SubItem{One of the Network Engineers is a Chief Security Officer(CSO). List of required certificates: CCISO, CCNA Security, ISO27001}
    \item Approx. 2 Help desk support (With some turnaround, as they are mostly apprentices.)

\end{itemize}

\hfill

\emph{Tier 1} is reserved for the "important" people in the business, and includes all upper management and regional managers.
\setlist{nolistsep}
\begin{itemize}[noitemsep]
    \item CEO - Chief Executive Officer
    \item CFO - Chief Financial Officer
    \item CTO - Chief Technology Officer (also Chief Data Officer)
    \item One Regional Managers per sub-branch.
    \item All department managers.
    \item One Financial Manager per location.
\end{itemize}

\hfill

\emph{Tier 2} is for employees that need more access than tier 1. This tier includes financial, HR and marketing.
\setlist{nolistsep}
\begin{itemize}[noitemsep]
    \item Financial
        \SubItem{2 Salary Consultants}
        \SubItem{2 Acquisition Managers}
        \SubItem{Financial manager per department}
        \SubItem{4 Accountants}
    \item Human resources
        \SubItem{4 Employees}
    \item Marketing
        \SubItem{Graphic Designer}
\end{itemize}

\hfill

\emph{Tier 3} is for all lower level employees that only need access to basic services on the network. This tier includes the vast amount of the employees of the business, with all the drivers, receptionists, the booking centre and mechanics.
\setlist{nolistsep}
\begin{itemize}
    \item Drivers
    \item Receptionists
        \SubItem{Receptionist at each location.}
    \item Booking Center
        \SubItem{2-10 Call Center Employees}
    \item Garage
        \SubItem{3 Logistics Managers}
        \SubItem{Head Mechanic per location}
        \SubItem{20 Mechanics}
\end{itemize}

\emph{IP Phones} are in their own group, even though most employees have an IP Phone, because the phones are placed in their own VLAN on the network.

