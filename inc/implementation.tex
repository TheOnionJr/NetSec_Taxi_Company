\chapter{Implementation}
\label{chap:implementation}

\section{Topology}
Our network design utilises a redundant core architecture in the HQ.
This is done to preserve network operation in the event of a link or device failure.
Each branch office has an identical layout of two access layer switches and one router.
These branch offices are connected to HQ via GRE tunnels.

Refer to appendix \todo{add appendix} for complete network topology.

\section{Redundancy}
All access layer switches have two links to the routing layer. 
Each of these links go to a different routing switch which in turn has a minimum of two links to another routing device (Router or L3 switch).
The router in the branch office does not have a redundant link to another routing device as the branch offices have a high MTPD.

The HQ also has two edge routers that provide a redundant link to the Internet and the branch offices in order to achieve the low MTPD requirement as specified in the requirements \ref{sec:reqirements}.

Link redundancy, load-balancing and loop prevention is handled by EIGRP.

\section{WLAN/WiFi}
All LightWeight Access Points (LWAP) are controlled by a single controller connected to B1-S3.
This does give a single point of failure, but WiFi is not considered business critical and therefore it is more cost efficient to just use one controller.
The LWAPs are spread out to multiple areas and it is estimated that four LWAPs pr area should be sufficient to cover Mjøstaxi's needs.