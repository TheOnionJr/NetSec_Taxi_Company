\chapter{Technology} % A different title might be more suitable?
\label{chap:technology}

% Chapter describing the technologies required for the suggested network.

This chapter explains the different technologies needed to realize the project.

\section{Servers}
The company has opted to host their servers on-site.
All the servers will need to be reachable from the internet, and is therefore put in a DMZ at the edge of the network in the HQ location Gjovik.

\subsubsection{Web Server}
MjøsTaxi wants to allow customers to order rides through the internet, rather than only having the option of calling the call-center to order a ride.

The company has hired a separate company to design a web-application, which will need to run on a web server.

\subsubsection{Internal Application Server}
The company has a local intranet application under development. This application will be hosted on an internal application server cluster in the HQ, and will be reachable through the web application for employees both located in the HQ, and from the outside. %We might want to look into this regarding traffic permissions from outside the intranet.

\subsubsection{E-mail Server}
The company's email will be handled by email servers located in each branch.
This server will have a relay in the DMZ, filtering and routing traffic to the main email server.



\section{VoIP}
MjøsTaxi is moving all its landlines over to VoIP.
Voice Over Internet Protocol relies on a VoIP service provider, meaning most of the work is being done offsite, with a provider that MjøsTaxi has a contract with. \cite{VoIP} 
The only setup needed on MjøsTaxis end, is making sure the network is set up such that the VoIP phones can reach the internet, to communicate with the VoIP provider, as well as setting up QoS.

\section{Networking}
\subsection{Hardware}
%This is probably redundant filler, ngl. Cut it if you feel like it.
To be able to provide sufficient network service the hardware required is:
 \setlist{nolistsep}
    \begin{itemize}[noitemsep]  
        \item Routers
        \item Switches
        \item Wireless Access-points
    \end{itemize}
Cisco hardware will be used for this implementation.    


\subsection{Mobile Networking}
Each taxi is equipped with a router connected to a mobile network through a third-party network provider.
This allows the taxis to communicate with the HQ, as well as payment services for the payment terminals.

\subsection{Protocols}

\subsubsection{DHCP}
Dynamic Host Configuration Protocol is a network management protocol.
The protocol dynamically assigns unique IP addresses to devices in the network.

\subsubsection{NAT}
Network Address Translation translates private IP addresses to legal addresses used on the internet, this allows for the creation of private internetworks. 
Essentially, entire networks can have the same IP address externally.

\subsubsection{VLAN}
Virtual Local Area Network protocol.
VLANs define broadcast domains in a layer 2 network, \cite{VLAN} in our case this can be used to confine network access for certain groups of users.
% Elaborate enough? 

\subsubsection{VTP}
VLAN Trunking Protocol.
This Cisco proprietary protocol propagates the definition of VLAN on the LAN.\cite{VTP} 
Using this protocol simplifies network configuration, by sharing VLANs across devices, limiting manual VLAN configuration to only a few devices.
%Too informal?

\subsubsection{EIGRP}
Enhanced Interior Gateway Routing Protocol is a Cisco-proprietary distance-vector routing protocol for automating routing decisions and configuration across a network. \cite{EIGRP}

\subsubsection{EtherChannel}
EtherChannel is a port link aggregation technology. EtherChannel allows grouping of several physical Ethernet links to create one logical link for redundancy and/or increased speed between two devices. \cite{EtherChannel}

\subsubsection{GRE Tunnel}
\vspace{-0.8em}
Generic Routing Encapsulation is a tunneling protocol developed by Cisco, capable of encapsulating a wide variety of protocol packet types in IP tunnels. \cite{CCNP-ROUTE}
These tunnels can be used to "extend" the network, essentially creating a link over the internet to another site.
GRE alone does not include any strong security mechanisms protecting its payload, however in conjunction with IPsec it can effectively be turned into a VPN.

\subsubsection{IPsec}
\vspace{-0.8em}
To secure the GRE tunnel, IPsec transport mode is used.
Transport mode only sees traffic going through an interface, and either encrypts or decrypts it. 
The combination of GRE tunnel and IPsec encryption effectively creates a VPN connection between sites in a cost-effective way.

\subsubsection{ICMP}
\vspace{-0.8em}
Internet Control Message Protocol is an internet-layer supporting protocol.
ICMP provides message packets to report errors and other IP packet processing information back to the source. \cite{ICMP}

\subsection{Security}

\subsubsection{TACACS+}
Terminal Access Controller Access-Control System Plus is a protocol that handles authentication, authorization and accounting services (AAA).
TACACS+ separates the AAA elements, allowing for more flexibility compared to e.g. RADIUS.
\cite{TACACS+}

\subsubsection{ACL}
Access Control Lists can be used to manage access to different parts of the network for different users or user groups.

\subsection{Standards}

\subsubsection{VoIP QoS}
VoIP Quality of Service is a concept required for VoIP to be a realistic replacement of the standard public switched telephone network telephones.
As VoIP is a real-time application, it is bandwidth- and delay-sensitive, QoS is required to ensure intelligible VoIP transmissions for both parties, through reduction of packet-loss, jitter and delay, by defining standards for services supply. \cite{VoIP-QoS}
